\documentclass[a4paper,11pt]{scrartcl}
\usepackage{amsmath} % provides \begin{align*}
\usepackage{outlines}
\usepackage[ngerman]{babel}
\usepackage[T1]{fontenc}
\usepackage[utf8]{inputenc}
\author{Tim Meusel}
\title{Mathe Hausaufgaben in \LaTeX{} von \author{}}
\date{\copyright\today}
%
\begin{document}
\maketitle

\section{Aufgabe}

$V(t) = 10t^3 - 50t^2 - 60t$

\section{Fragen}

\subsection{Wann befindet sich der Schlitten in Ruhe?}
Bei $V(t) = 0$ bewegt sich der Schlitten nicht. Die möglichen Positionen können
über die Nullstellen von V ermittelt werden. Vorgehensweise:

\begin{outline}
  \1 Formel nach $V(t) = 0$ umstellen
  \1 In eine quadratische Funktion umwandeln, welche 0 ergibt
  \1 Faktor vor der Variable mit dem Exponent 2 entfernen
  \1 PQ Formel anwenden
\end{outline}

Realisierung:
\begin{align*}
  V(t) &= 10t^3 - 50t^2 - 60t                   \\
  V(t) &= 0 <=> 10t^3 - 50t^2 - 60t = 0         \\
  V(t) &= 0 <=> t * (10t^2 - 50t - 60) = 0      \\
  V(t) &= 0 <=> t = 0 \vee 10t^2 - 50t + 60 = 0 \\
     0 &= 10t^2 - 50t + 60                      \\
     0 &= t^2 - 5t + 6                          \\
 x_1,_2 &= -\frac{p}{2} \pm \sqrt{\frac{p}{2}^2 - s}\\
\end{align*}

Da $V(t) = 0 <=> t = 0 \vee 10t^2 - 50t + 60 = 0$ ist eine Nullstelle bei $x = 0$.

\subsection{Wann ist die größte und die kleinste Geschwindigkeit im Intervall
erreicht und wie groß sind diese?}

\subsection{Die maximale Beschleunigung beträgt $20cm / minute^2$. Wird dieser
Wert im Intervall [1;2] min überschritten?}

\subsection{Welcher Weg wird im Intervall [0;3,4] zurückgelegt?}


\end{document}
