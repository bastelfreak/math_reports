\documentclass[a4paper,11pt]{scrartcl}
\usepackage{amsmath} % provides \begin{align*}
\usepackage{outlines}
\usepackage[ngerman]{babel}
\usepackage[T1]{fontenc}
\usepackage[utf8]{inputenc}
\author{Tim Meusel}
\title{Mathe Hausaufgaben in \LaTeX{}}
\date{\copyright\today}
%
\begin{document}
\maketitle

\section{Aufgabe}

$V(t) = 10t^3 - 50t^2 - 60t$

\section{Fragen}

\subsection{Wann befindet sich der Schlitten in Ruhe?}
Bei $V(t) = 0$ bewegt sich der Schlitten nicht. Die möglichen Positionen können
über die Nullstellen von V ermittelt werden. Vorgehensweise:

\begin{outline}
  \1 Formel nach $V(t) = 0$ umstellen
  \1 In eine quadratische Funktion umwandeln, welche 0 ergibt
  \1 Faktor vor der Variable mit dem Exponent 2 entfernen
  \1 PQ Formel anwenden
  \1 Ausrechnen
\end{outline}

Realisierung:
\begin{align*}
  V(t) &= 10t^3 - 50t^2 - 60t                               \\
  V(t) &= 0 \Leftrightarrow 10t^3 - 50t^2 - 60t = 0         \\
  V(t) &= 0 \Leftrightarrow t * (10t^2 - 50t - 60) = 0      \\
  V(t) &= 0 \Leftrightarrow t = 0 \vee 10t^2 - 50t + 60 = 0 \\
     0 &= 10t^2 - 50t + 60                                  \\
     0 &= t^2 - 5t + 6                                      \\
 x_1,_2 &= -\frac{p}{2} \pm \sqrt{\frac{p}{2}^2 - q}        \\
 x_1,_2 &= -\frac{5}{2} \pm \sqrt{\frac{5}{2}^2 - 6}        \\
 x_1,_2 &= 2,5 \pm 0,5                                      \\
 x_1    &= 3                                                \\
 x_2    &= 2                                                \\
\end{align*}

Da $V(t) = 0 \Leftrightarrow t = 0 \vee 10t^2 - 50t + 60 = 0$ ist eine
Nullstelle bei $x = 0$. Die weiteren Nullstellen befinden sich bei $x = 3$ und
$x = 2$.

\subsection{Wann ist die grö{\ss}te und die kleinste Geschwindigkeit im
Intervall erreicht und wie gro{\ss} sind diese?}

\begin{outline}
  \1 Ableiten
  \1 Faktor vor der quadriertn Variable entfernen
  \1 PQ Formel anwenden
  \1 Ausrechnen
\end{outline}

\begin{align*}
  V(t) &= 10t^3 - 50t^2 - 60t                               \\
  V^\prime(t) &= 30t^2 - 100t + 60  \\
  V^\prime(t) &= t^2  - 4\frac{1}{3}t + 2  \\
  x_1,_2 &= -\frac{p}{2} \pm \sqrt{\left(\frac{p}{2}\right)^2 - q} \\
  x_1,_2 &= \frac{\frac{1}{3}}{2} \pm \sqrt{\left(\frac{-1\frac{1}{3}}{2}\right)^2 - 2} \\
  x_1,_2 &= \frac{\frac{1}{3}}{2} \pm \frac{{\sqrt{7}}}{3} \\
  x_1   &= 1,0485  \\
  x_1   &= -0,7152  \\
\end{align*}

\begin{align*}
  V^{\prime\prime}(t) &= 60t - 100      \\
  V^{\prime\prime}(-0,7152) &= -142,912 \\
  V^{\prime\prime}(1,0485) &= 60,385    \\
\end{align*}

\subsection{Die maximale Beschleunigung beträgt $20cm / minute^2$. Wird dieser
Wert im Intervall [1;2] min überschritten?}

\subsection{Welcher Weg wird im Intervall [0;3,4] zurückgelegt?}


\end{document}
